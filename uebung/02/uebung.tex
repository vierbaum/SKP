\documentclass{article}
\usepackage{tikz}
\usepackage{pgfplots}
\usepackage{amsmath, amsfonts}
\usepackage{enumerate}
\usepackage{pdfpages}
\input{title.tex}

\newcommand{\N}{\mathbb{N}}
\newcommand{\R}{\mathbb{R}}
\newcommand{\E}{\mathbb{E}}
\newcommand{\V}{\mathbb{V}}
\renewcommand{\P}{\mathbb{P}}
\begin{document}
    \maketitle
    \section{}
    \section{}
    \begin{enumerate}[a)]
        \item
            \begin{enumerate}[i.]
                \item
                    \begin{tabular}{|l|c|}
                        \hline
                        Schritt:&\\
                        \hline
                        \hline
                        Prozessor führt i-ten Befehl des Benutzerprogramms aus. Währenddes-&a\\
                        sen wird ein Interrupt an den UBR-Controller gemeldet. Der UBR-&\\
                        Controller legt das entsprechende Unterbrechungssignal an den Prozes-&\\
                        sor an.&\\
                        \hline
                        Prozessor führt den i-ten Befehl zu Ende aus.&d\\
                        \hline
                        Prozessor überprüft, ob ein Unterbrechungssignal vom UBR-Controller&f\\
                        vorliegt.&\\
                        \hline
                        Prozessor unterbricht Ausführung des Benutzerprogramms und beginnt&e\\
                        Unterbrechungsbehandlung. Der Prozessor signalisiert dies dem UBR-&\\
                        Controller. Der Zustand des Benutzerprogramms wird gesichert.&\\
                        \hline
                        Prozessor lädt neuen Befehl entsprechend der Unterbrechungsvektorta-&g\\
                        belle.&\\
                        \hline
                        Prozessor führt Unterbrechungsroutine aus. Die Unterbrechungsroutine&b\\
                        wird ausgeführt. Dann wird die Unterbrechungsroutine verlassen.&\\
                        \hline
                        Der Zustand des Benutzerprogramms wird wiederhergestellt.&c\\
                        \hline
                        Prozessor lädt den (i+1)-ten Befehl des Benutzerprogramms und führt&h\\
                        diesen aus.&\\
                        \hline
                    \end{tabular}
                \item
                    \begin{tabular}{|l|c|}
                        \hline
                        Schritt:&\\
                        \hline
                        \hline
                        Prozessor führt i-ten Befehl des Benutzerprogramms aus. Währenddes-&a\\
                        sen wird ein Interrupt an den UBR-Controller gemeldet. Der UBR-&\\
                        Controller legt das entsprechende Unterbrechungssignal an den Prozes-&\\
                        sor an.&\\
                        \hline
                        Prozessor führt den i-ten Befehl zu Ende aus.&d\\
                        \hline
                        Prozessor überprüft, ob ein Unterbrechungssignal vom UBR-Controller&f\\
                        vorliegt.&\\
                        \hline
                        Prozessor lädt den (i+1)-ten Befehl des Benutzerprogramms und führt&h\\
                        diesen aus.&\\
                        \hline
                    \end{tabular}
            \end{enumerate}
        \item
            \begin{tabular}{|c|c|c|}
                \hline
                &In PCB enthalten&Nicht in PCB enthalten\\
                \hline
                \hline
                Prozessidentität (pid)&$\times$&\\
                \hline
                Maschinencode des Programms&&$\times$\\
                \hline
                Programmcounter, bzw. Befehlszeiger&$\times$&\\
                \hline
                Kopie der Systemlibrary&&$\times$\\
                \hline
                Globale Variablen des Prozesses&&$\times$\\
                \hline
                Zeiger auf Prozessspeicherbereiche&$\times$&\\
                \hline
            \end{tabular}
        \item
            Wir haben N(eu), Be(reit), A(ktiv), Bl(ockiert), T(erminiert)\\
            \begin{tabular}{|c|c|c|c|c|c|c|c|c|c|c|}
                \hline
                \textbf{Prozess}&$t_0$&$t_1$&$t_2$&$t_3$&$t_4$&$t_5$&$t_6$&$t_7$&$t_8$&$t_9$\\
                \hline
                p&N&A&Bl&Bl&A&Be&A&T&T&T\\
                \hline
                q&N&Be&A&Be&Be&A&Bl&A&T&T\\
                \hline
                r&N&Be&Be&A&Bl&Bl&Be&Be&A&T\\
                \hline
            \end{tabular}
    \end{enumerate}
\end{document}
