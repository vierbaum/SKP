\documentclass{article}
\usepackage{tikz}
\usepackage{pgfplots}
\usepackage{amsmath, amsfonts}
\usepackage{enumerate}
\input{title.tex}

\newcommand{\N}{\mathbb{N}}
\newcommand{\R}{\mathbb{R}}
\newcommand{\E}{\mathbb{E}}
\newcommand{\V}{\mathbb{V}}
\renewcommand{\P}{\mathbb{P}}
\begin{document}
    \maketitle
    \section{}
    \section{}
    \begin{enumerate}[a)]
        \item
            \begin{enumerate}[i.]
                \item
                    Registermaschine:\\
                    \begin{tabular}{l|l}
                        \multicolumn{2}{l}{
                            $\pi$
                        }\\
                        \hline
                        \multicolumn{2}{l}{
                            load R1, 0x00
                        }\\
                        \multicolumn{2}{l}{
                            store R1, n
                        }\\
                        \hline
                        p:&q:\\
                        load R1, n&load R1, n\\
                        add R1, 0x01&sub R1, 0x01\\
                        store R1, n&store R1, n
                    \end{tabular}\\
                    Stapelmaschine:\\
                    \begin{tabular}{l|l}
                        \multicolumn{2}{l}{
                            $\pi$
                        }\\
                        \hline
                        \multicolumn{2}{l}{
                            push 0
                        }\\
                        \multicolumn{2}{l}{
                            pop n
                        }\\
                        \hline
                        p:&q:\\
                        push n&push n\\
                        push 1&push1\\
                        add&sub\\
                        pop n&pop n
                    \end{tabular}
                \item
                    Korrekter Ablauf $n=0$ zum Ende.
                    Registermaschine:\\
                    \begin{tabular}{cl|l}
                        thread&Programm & n\\
                        \hline
                        &load R1, 0x00&UNDEF\\
                        &store R1, n&0\\
                        p&load R1, n&0\\
                        p&add R1, 0x01&0\\
                        p&store R1, n&1\\
                        q&load R1, n&1\\
                        q&sub R1, 0x01&1\\
                        q&store R1, n&0\\
                    \end{tabular}
                    Stapelmaschine:
                    \begin{tabular}{cl|l}
                        thread&Programm & n\\
                        \hline
                        &push 0&UNDEF\\
                        &pop n&0\\
                        p&push n&0\\
                        p&push 1&0\\
                        p&add&0\\
                        p&pop n&1\\
                        q&push n&1\\
                        q&push 1&1\\
                        q&sub&1\\
                        q&pop n&0\\
                    \end{tabular}\\
                    Inkorrekter Ablauf $(n\neq 0)$ zum Ende.
                    Registermaschine:\\
                    \begin{tabular}{cl|l}
                        thread&Programm & n\\
                        \hline
                        &load R1, 0x00&UNDEF\\
                        &store R1, n&0\\
                        p&load R1, n&0\\
                        q&load R1, n&0\\

                        q&sub R1, 0x01&0\\
                        p&add R1, 0x01&0\\
                        q&store R1, n&-1\\
                        p&store R1, n&1\\
                    \end{tabular}
                    Stapelmaschine:
                    \begin{tabular}{cl|l}
                        thread&Programm & n\\
                        \hline
                        &push 0&UNDEF\\
                        &pop n&0\\
                        p&push n&0\\
                        p&push 1&0\\
                        p&add&0\\
                        q&push n&0\\
                        q&push 1&0\\
                        q&sub&0\\
                        q&pop n&-1\\
                        p&pop n&1\\
                    \end{tabular}
            \end{enumerate}
        \item
            $\{4, 8, 12,\hdots,40\}$
        \item
            \begin{enumerate}[i.]
                \item $(8, 0)$
                \item $(8, 0)$
            \end{enumerate}
    \end{enumerate}
\end{document}
